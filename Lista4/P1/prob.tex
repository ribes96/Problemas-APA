\documentclass[a4paper]{article}
\usepackage[utf8]{inputenc}
\usepackage[spanish]{babel}

\author{Albert Ribes}
\title{Problema 1}

\begin{document}
  \maketitle
  Considerem un problema de classificació en dues classes, en les quals es disposa de les probabilitats de
  cada classe $P(C_1)$ i $P(C_2)$. Considerem tres possibles regles per classificar un objecte:
  \begin{enumerate}
    \item ($R_1$) Predir la classe més probable
    \item ($R_2$) Predir la classe $C_1$ amb probabilitat $P(C_1)$
    \item ($R_3$) Predir la classe $C_1$ amb probabilitat $0.5$
  \end{enumerate}
  Es demana:
  \begin{enumerate}
    \item Donar les probabilitats d'error $P_i(error)$ de les tres regles, $i = 1, 2, 3$
    {\bfseries
    \begin{itemize}
      \item El error de la regla $R_1$ es $min(P(C_1), P(C_2))$
      \item Sea $Q(C_i)$ la probabilidad de predecir la clase $C_i$. La probabilidad de error de la regla $R_2$ es $P(C_1) \wedge Q(C_2) + P(C_2) \wedge Q(C_1)$. Puesto que $P$ y $Q$ son probabilidades
      independientes, se puede escribir como $P(C_1) \cdot Q(C_2) + P(C_2) \cdot Q(C_1)$. Pero $Q(C_i) = P(C_i)$, como indica la regla. Por lo tanto el error de la regla $R_2$ es $P(C_1) \cdot (1 - P(C_1)) + (1 - P(C_1)) \cdot P(C_1)$, que equivale a $2P(C_1) - 2P(C_1)^2$
      \item Este es un caso particular de la regla $R_2$. Ahora el error es
      $P(C_1) \cdot 0,5 + (1 - P(C_1)) \cdot 0,5 \equiv 0,5$
    \end{itemize}

    }
    \item Demostrar que $P_1(error) \leq P_2(error) \leq P_3(error)$

    {\bfseries

    Sin pérdida de generalidad asumiremos que $P(C_1) \geq \frac{1}{2}$. El caso
    contrario es simétrico. Entonces para la primera parte de la demostración
    hay que demostrar que
    \begin{eqnarray*}
    P(C_2)  \leq 2P(C_1) - 2P(C_1)^2  & \equiv\\
    1 - P(C_1)  \leq 2P(C_1) - 2P(C_1)^2 & \equiv \\
    2P(C_1)^2 - 3P(C_1) + 1  \leq 0  &
    \end{eqnarray*}

    Usaremos la fórmula de las ecuaciones de segundo grado para resolverlo:

    \begin{eqnarray*}
      P(C_1) = \frac{3 \pm \sqrt{9 - 4 \cdot 2 \cdot 1}}{2 \cdot 2} = \frac{3 \pm 1}{4}
      \Rightarrow P(C_1) \notin (\frac{1}{2} , 1)
    \end{eqnarray*}

    Esto es claramente una contradicción. Habrá que ver qué nos está pasando





    }
  \end{enumerate}
\end{document}
