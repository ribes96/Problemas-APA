\documentclass[a4paper]{article}
\usepackage[utf8]{inputenc}
\usepackage[catalan]{babel}
\usepackage{mathtools}
\usepackage{amsmath}
\usepackage{multicol}


\author{Albert Ribes}
\title{Problema 4: Propietats elàstiques d'una molla [R]}

\begin{document}
\maketitle
Volem determinar les propietats elàstiques d'una molla usant diferents pesos i mesurant la deformació
que es produeix. La llei de Hooke relaciona la longitud l i la força F que exerceix el pes com:

\begin{equation*}
e + kF = l
\end{equation*}

on $e$, $k$ són constants de la llei, que es volen determinar. S'ha realitzat un experiment i obtingut les
dades:

\begin{center}
\begin{tabular}{c c c c c c}
  F & 1 & 2 & 3 & 4 & 5 \\
  \hline
  l & 7.97 & 10.2 & 14.2 & 16.0 & 21.2 \\
\end{tabular}
\end{center}

\begin{enumerate}
  \item Plantegeu el problema com un problema de mínims quadrats

  {\bfseries

  La ecuación que plantearemos es:

  \begin{equation*}
    l = y(F; w) = w_0\phi_0(F) + w_1\phi_1(F)
  \end{equation*}

  De modo que $w_0$ será $e$ y $w_1$ será $k$. Para poder establecer la equivalencia con la fórmula indicada hay que elegir unas funciones de base en particular.
  Definimos la variable objetivo ($l$), los datos de entrada ($F$), y las funciones de base ($\phi_0$, $\phi_1$, $\phi$ y $\Phi$)
  \begin{multicols}{3}

    \begin{equation*}
      l =
      \begin{bmatrix}
    7.97 \\
    10.2 \\
    14.2 \\
    16.0 \\
    21.2 \\
      \end{bmatrix}
    \end{equation*}

    %%%%%%%%

    \begin{equation*}
      F =
      \begin{bmatrix}
        1 \\
        2 \\
        3 \\
        4 \\
        5 \\
      \end{bmatrix}
    \end{equation*}

    \begin{equation*}
      \phi_0(x) = 1
    \end{equation*}

    \begin{equation*}
      \phi_1(x) = x
    \end{equation*}

    \begin{equation*}
      \phi =
      \begin{bmatrix}
        1 \\
        x \\
      \end{bmatrix}
    \end{equation*}

    \begin{equation*}
      \Phi =
      \begin{bmatrix}
        1 & 1 \\
        1 & 2\\
        1 & 3 \\
        1 & 4 \\
        1 & 5 \\
      \end{bmatrix}
    \end{equation*}

  \end{multicols}


Y ahora con estos datos habría que encontrar $w$ mediante la resolución de
la ecuación:

\begin{equation*}
  \Phi^T\Phi w = \Phi^T l
\end{equation*}

}


  \item Resoleu-lo amb el métode de la matriu pseudo-inversa

{\bfseries

  Hay que resolver:

  \begin{equation*}
    w = (\Phi^T\Phi)^{-1}\Phi^T l
  \end{equation*}

  Si ponemos los datos en R, dice que:

\begin{multicols}{2}

  \begin{equation*}
    \Phi^T\Phi =
    \begin{bmatrix}
        5 & 15 \\
        15 & 55
    \end{bmatrix}
  \end{equation*}

  \begin{equation*}
    (\Phi^T\Phi)^{-1} =
    \begin{bmatrix}
        1.1 & -0.3 \\
        -0.3 & 0.1
    \end{bmatrix}
  \end{equation*}


  \begin{equation*}
    (\Phi^T\Phi)^{-1} \Phi^T =
    \begin{bmatrix}
      0.8 & 0.5 & 0.2 & -0.1 & -0.4 \\
      -0.2 & -0.1 & 0.0 & 0.1 & 0.2
    \end{bmatrix}
  \end{equation*}

  \begin{equation*}
    (\Phi^T\Phi)^{-1} \Phi^T l =
    \begin{bmatrix}
      4.236 \\
      3.226
    \end{bmatrix}
  \end{equation*}


\end{multicols}

De modo que $e = w_0 = 4.236$ y $k = w_1 = 3.226$

}


  \item Resoleu-lo amb el métode basat en la SVD

  {\bfseries

  Si llamamos al método \texttt{svd()} de R con la matriz de diseño $\Phi$ nos
  retorna:

  \begin{multicols}{2}
    \begin{equation*}
      U =
      \begin{bmatrix}
        0.1600071 & 0.7578903 \\
         0.2853078 & 0.4675462 \\
        0.4106086 & 0.1772020 \\
         0.5359094 & -0.1131421 \\
        0.6612102 & -0.4034862
      \end{bmatrix}
    \end{equation*}





    \begin{equation*}
      V =
      \begin{bmatrix}
        0.2669336 & 0.9637149 \\
        0.9637149 & -0.2669336
      \end{bmatrix}
    \end{equation*}


    \begin{equation*}
      \Delta =
      \begin{bmatrix}
        7.691213 & 0.0000000 \\
        0.000000 & 0.9193696
      \end{bmatrix}
    \end{equation*}

  \end{multicols}

  De modo que tenemos que resolver:
  \begin{equation*}
    w = (\Phi^T\Phi)^{-1}\Phi^T l = ((U\Delta V)^T(U\Delta V))^{-1}(U\Delta V)^T l
  \end{equation*}

  \begin{equation*}
    w = V\Delta^{-1}U^T l
  \end{equation*}

  \begin{equation*}
    w =
    \begin{bmatrix}
      0.8 & 0.5 & 2.000000e-01 & -0.1 & -0.4 \\
      -0.2 & -0.1 & 2.081668e-17 & 0.1 & 0.2
    \end{bmatrix}
      l
  \end{equation*}

  \begin{equation*}
    w =
    \begin{bmatrix}
      4.236 \\
      3.226
    \end{bmatrix}
  \end{equation*}


  Se llega a la misma conclusión que en el apartado anterior, puesto que
  en este problema la resolución de la inversa directamente no daba
  demasiados problemas.

  }



\end{enumerate}




\end{document}
