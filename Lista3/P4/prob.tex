\documentclass[a4paper]{article}
\usepackage[utf8]{inputenc}
\usepackage[catalan]{babel}
\usepackage{mathtools}
\usepackage{amsmath}


\author{Albert Ribes}
\title{Problema 4: Propietats elàstiques d'una molla [R]}

\begin{document}
\maketitle
Volem determinar les propietats elàstiques d'una molla usant diferents pesos i mesurant la deformació
que es produeix. La llei de Hooke relaciona la longitud l i la força F que exerceix el pes com:

\begin{equation*}
e + kF = l
\end{equation*}

on $e$, $k$ són constants de la llei, que es volen determinar. S'ha realitzat un experiment i obtingut les
dades:

\begin{center}
\begin{tabular}{c c c c c c}
  F & 1 & 2 & 3 & 4 & 5 \\
  \hline
  l & 7.97 & 10.2 & 14.2 & 16.0 & 21.2 \\
\end{tabular}
\end{center}

\begin{enumerate}
  \item Plantegeu el problema com un problema de mínims quadrats

  La variable objetivo será:

\begin{equation*}
  t =
  \begin{bmatrix}
7.97 \\
10.2 \\
14.2 \\
16.0 \\
21.2 \\
  \end{bmatrix}
\end{equation*}

Los datos de entrada serán:

\begin{equation*}
  X =
  \begin{bmatrix}
    1 \\
    2 \\
    3 \\
    4 \\
    5 \\
  \end{bmatrix}
\end{equation*}

Y las funciones de base serán:

\begin{equation*}
  \phi_0(x) = e
\end{equation*}

\begin{equation*}
  \phi_1(x) = 1
\end{equation*}

De modo que:

\begin{equation*}
  \phi =
  \begin{bmatrix}
    e \\
    1 \\
  \end{bmatrix}
\end{equation*}


Y:

\begin{equation*}
  \Phi =
  \begin{bmatrix}
    e & 1 \\
    e & 1\\
    e & 1 \\
    e & 1 \\
    e & 1 \\
  \end{bmatrix}
\end{equation*}


Y ahora con estos datos habría que encontrar $w$ mediante la resolución de
la ecuación:

\begin{equation*}
  \Phi^T\Phi w = \Phi^T t
\end{equation*}


  \item Resoleu-lo amb el métode de la matriu pseudo-inversa
  \item Resoleu-lo amb el métode basat en la SVD
\end{enumerate}



\end{document}
