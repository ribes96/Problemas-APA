\documentclass[a4paper]{article}
\usepackage[utf8]{inputenc}
\usepackage[spanish]{babel}
\usepackage[svgnames]{xcolor}
\usepackage{listings}

\author{
Gerard Barrachina
\and
Josep de Cid
\and
Albert Ribes
\and
Kerstin Winter
}

\title{Problema 7: Reconeixement de lletres amb la xarxa MLP [R,G]}


\lstset{language=R,
    basicstyle=\small\ttfamily,
    stringstyle=\color{DarkGreen},
    otherkeywords={0,1,2,3,4,5,6,7,8,9},
    morekeywords={TRUE,FALSE},
    deletekeywords={data,frame,length,as,character},
    keywordstyle=\color{blue},
    commentstyle=\color{DarkGreen},
}


\begin{document}
    \maketitle

    \begin{enumerate}
        \item Dissenyeu una funció que generi versions corruptes d'una lletra, a còpia de canviar un cert número
        de bits de manera aleatòria. Una manera senzilla és generar primer el número de bits corruptes p.e.
        amb una Poisson ($\lambda = 1.01$)
        seleccionar els bits concrets (uniformement) i després invertir-los.

        \begin{lstlisting}
        corrupt = function(bit_vector, nchanges) {
changes = sample(length(bit_vector), nchanges)
g = rep(0, length(bit_vector))
for (n in changes) {
g[n] = 1
}
new = as.numeric(xor(bit_vector, g))
return(new)
}
        \end{lstlisting}

        \item Dissenyeu una funció que, partint de les lletres netes (arxiu
        corruptes que usarem com a mostra de training, de mida
        letters.txt),
        generi unes dades
        $N$.

        \begin{lstlisting}
        generate_corrupt = function(df, n) {
          nm = c(1:35, "letter")
          g = sample(1:nrow(df), n, replace = TRUE)
          ch = rpois(n, 1.01)
          new_data = data.frame()
          for (i in 1:length(g)) {
            elem = g[i]
            new_vector = corrupt(df[elem, -36], ch[i])
            newrow = c(new_vector, df[elem, 36])
            new_data = rbind(new_data, newrow)
            new_data[i,36] = as.character(df[elem, 36])
          }
          colnames(new_data) = nm
          new_data$letter = as.factor(new_data$letter)
          return(new_data)
        }
        \end{lstlisting}

        \item Entreneu una xarxa MLP per aprendre la tasca. Caldrà que estimeu la millor arquitectura, cosa
        que podeu fer per cross-validation, usant regularització.
        \begin{lstlisting}
        train_data = generate_corrupt(mydata, 1000)
test_data = generate_corrupt(mydata, 300)

trc <- trainControl (method="repeatedcv", number=10, repeats=5)
## WARNING: this takes maaaaaany minutes
caret.nnet.model <- train (
  letter ~.,
  data = train_data,
  method='nnet',
  metric = "Accuracy",
  trControl=trc)

test_results = predict(caret.nnet.model, test_data)

confusionMatrix(test_data$letter, test_results)
        \end{lstlisting}
        \item Reporteu els resultats de predicció en una mostra de test gran -també generada per vosaltres, i de
        manera anàloga a la de training.

        Usando repeatedcv hemos visto que la mejor red tine 5 neuronas, haciendo regulación con $decay = 0.1$.

        Hemos conseguido una accuracy de $0.93$

    \end{enumerate}

\end{document}
