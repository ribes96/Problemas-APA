\documentclass[a4paper]{article}
\usepackage[utf8]{inputenc}
\usepackage[spanish]{babel}
\usepackage{amsmath}
\usepackage{amsfonts}
\usepackage{listings}
\usepackage[svgnames]{xcolor}


\lstset{language=R,
    basicstyle=\small\ttfamily,
    stringstyle=\color{DarkGreen},
    otherkeywords={0,1,2,3,4,5,6,7,8,9},
    morekeywords={TRUE,FALSE},
    deletekeywords={data,frame,length,as,character},
    keywordstyle=\color{blue},
    commentstyle=\color{DarkGreen},
}

\author{Albert Ribes}
\title{\textbf{5. Pràctica amb la xarxa MLP 1 [R]}}

\begin{document}
    \maketitle

    Aquesta és una tasca usada com a benchmark en la literatura. Definim
    $f:
    [-1, 1]^2
    \longmapsto
    \mathbb{R} $
    com:
    $f(x_1, x_2) =
    4\sin(\pi x_1) +
    2\cos(\pi x_2 ) +
    \epsilon$
    on
    $\epsilon \sim \mathcal{N}(0, 0.5^2)$
    és soroll normal amb mitjana zero i desviació estàndar
    $1$.

    \begin{enumerate}

        \item Entreneu una xarxa neuronal MLP amb la rutina
        \texttt{nnet\{nnet\}}
        per aprendre la tasca. Heu de fer $4$
        estudis separats, prenent conjunts d'aprenentatge de mida creixent:
            $100$, $200$, $500$ i $1000$, mostrejats
        de manera aleatòria uniformement en
        $[-1, 1]^2$ .
        Caldrà que estimeu la millor arquitectura, cosa que
        podeu fer per \textit{cross-validation}, usant regularització.

        \begin{lstlisting}
        trc <- trainControl (
            method="repeatedcv",
            number=10,
            repeats=5)
        model1 <- train (
            target ~.,
            data = df1,
            linout = TRUE,
            method='nnet',
            metric = "RMSE",
            trControl=trc)
        \end{lstlisting}

        {\bfseries
        Se han generado 4 dataframes con distintos tamaños (\texttt{df1, df2, df3, df4}) y se ha usado \texttt{caret} para encontrar la arquitectura más adecuada para cada uno de ellos. Haciendo $5$ veces 10-fold-cross-validation se han generado 4 modelos distintos, uno para cada dataset de \textit{training}
        }

        \item Reporteu els resultats de predicció dels 4 estudis en un conjunt de test de mida $1024$
        obtingut de
        crear exemples a intervals regulars en
        $[-1, 1]^2$.

        \begin{table}[!htbp] \centering
  \caption{Resultados de testing}
  \label{}
\begin{tabular}{@{\extracolsep{5pt}} cccccc}
\\[-1.8ex]\hline
\hline \\[-1.8ex]
 & RMSE & Rsquared & MAE & size & decay \\
\hline \\[-1.8ex]
model1 & 0.403850910943991 & 0.985033338996773 & 0.317626232228504 & 9 & 0.1 \\
model2 & 0.281937902973326 & 0.992197331725038 & 0.227103601567067 & 10 & 0.1 \\
model3 & 0.273151994004781 & 0.992603513898941 & 0.216876595191761 & 8 & 0.1 \\
model4 & 0.3040618541533 & 0.990810885868111 & 0.240931949631532 & 9 & 0.1 \\
\hline \\[-1.8ex]
\end{tabular}
\end{table}

        \item Repetiu els experiments usant regressió lineal amb i sense regularització en els mateixos conjunts
        de dades i compareu els resultats obtinguts amb els de la xarxa MLP; noteu que podeu usar
        simplement la rutina
        \texttt{nnet}
        amb
        \texttt{size=0}.
    \end{enumerate}

\end{document}
