\documentclass[a4paper]{article}
\usepackage[utf8]{inputenc}
\usepackage[spanish]{babel}
\usepackage{amsmath}
\usepackage{amsfonts}

\author{Albert Ribes}
\title{\textbf{5. Pràctica amb la xarxa MLP 1 [R]}}

\begin{document}
    \maketitle

    Aquesta és una tasca usada com a benchmark en la literatura. Definim
    $f:
    [-1, 1]^2
    \longmapsto
    \mathbb{R} $
    com:
    $f(x_1, x_2) =
    4\sin(\pi x_1) +
    2\cos(\pi x_2 ) +
    \epsilon$
    on
    $\epsilon \sim \mathcal{N}(0, 0.5^2)$
    és soroll normal amb mitjana zero i desviació estàndar
    $1$.

    \begin{enumerate}

        \item Entreneu una xarxa neuronal MLP amb la rutina
        \texttt{nnet\{nnet\}}
        per aprendre la tasca. Heu de fer $4$
        estudis separats, prenent conjunts d'aprenentatge de mida creixent:
            $100$, $200$, $500$ i $1000$, mostrejats
        de manera aleatòria uniformement en
        $[-1, 1]^2$ .
        Caldrà que estimeu la millor arquitectura, cosa que
        podeu fer per \textit{cross-validation}, usant regularització.
        \item Reporteu els resultats de predicció dels 4 estudis en un conjunt de test de mida $1024$
        obtingut de
        crear exemples a intervals regulars en
        $[-1, 1]^2$.

        \item Repetiu els experiments usant regressió lineal amb i sense regularització en els mateixos conjunts
        de dades i compareu els resultats obtinguts amb els de la xarxa MLP; noteu que podeu usar
        simplement la rutina
        \texttt{nnet}
        amb
        \texttt{size=0}.
    \end{enumerate}

\end{document}
