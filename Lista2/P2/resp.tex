\documentclass[a4paper,10pt]{article}
\usepackage[utf8]{inputenc}
\usepackage[catalan]{babel}
\usepackage{amsmath}
\usepackage{amsfonts}
\usepackage{listings}
\usepackage[svgnames]{xcolor}
\usepackage{blkarray}

\lstset{language=R,
    basicstyle=\small\ttfamily,
    stringstyle=\color{DarkGreen},
    otherkeywords={0,1,2,3,4,5,6,7,8,9},
    morekeywords={TRUE,FALSE},
    deletekeywords={data,frame,length,as,character},
    keywordstyle=\color{blue},
    commentstyle=\color{DarkGreen},
}

%opening
\title{2. L'anàlisi de components principals en acció [G, R]}
\author{Albert Ribes}

\begin{document}

\maketitle

% \begin{abstract}
%
% \end{abstract}

% \section{}

\textbf{
Considerem un problema amb N = 8 dades bidimensionals:
}

\begin{equation*}
  \{(1, 2), (3, 3), (3, 5), (5, 4), (5, 6), (6, 5), (8, 7), (9, 8)\}
\end{equation*}

\begin{enumerate}
  \item \textbf{Calculeu la matriu de covariança mostral de les dades $\hat{\Sigma}$}
  \begin{lstlisting}[language=R]
  X1 = c(1,3,3,5,5,6,8,9)
  X2 = c(2,3,5,4,6,5,4,8)
  X = cbind(X1,X2) #Matriz de datos
  S = cov(X) #Matrix de covarianzas
\end{lstlisting}

Que da como resultado:

\begin{equation*}
\begin{bmatrix}
  7.142857 & 3.571429 \\
  3.571429 & 3.410714
\end{bmatrix}
\end{equation*}

  \item \textbf{Calculeu els dos valors propis de $\hat{\Sigma}$}
  \begin{lstlisting}[language=R]
  eigen(S, only.values = T)

  [1] 9.306342 1.247229
\end{lstlisting}




  \item \textbf{Calculeu els dos vectors propis corresponents a $a_1$ i $a_2$}

  \item \textbf{Dibuixeu les dades i les dues components principals}

  \item \textbf{Quin és el percentatge de variança explicada per la primera component principal?}
\end{enumerate}
\end{document}
