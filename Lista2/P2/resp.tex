\documentclass[a4paper,10pt]{article}
\usepackage[utf8]{inputenc}
\usepackage[catalan]{babel}
\usepackage{amsmath}
\usepackage{amsfonts}

%opening
\title{2. L'anàlisi de components principals en acció [G, R]}
\author{Albert Ribes}

\begin{document}

\maketitle

% \begin{abstract}
%
% \end{abstract}

% \section{}

\textbf{
Considerem un problema amb N = 8 dades bidimensionals:
}

\begin{equation*}
  \{(1, 2), (3, 3), (3, 5), (5, 4), (5, 6), (6, 5), (8, 7), (9, 8)\}
\end{equation*}

\begin{enumerate}
  \item \textbf{Calculeu la matriu de covariança mostral de les dades $\hat{\Sigma}$}



  \item \textbf{Calculeu els dos valors propis de $\hat{\Sigma}$}





  \item \textbf{Calculeu els dos vectors propis corresponents a $a_1$ i $a_2$}

  \item \textbf{Dibuixeu les dades i les dues components principals}

  \item \textbf{Quin és el percentatge de variança explicada per la primera component principal?}
\end{enumerate}
\end{document}
