\documentclass[a4paper,10pt]{article}
\usepackage[utf8]{inputenc}
\usepackage[catalan]{babel}
\usepackage{amsmath}
\usepackage{amsfonts}
\usepackage{listings}
\usepackage[svgnames]{xcolor}
\usepackage{blkarray}
\usepackage{breqn}

%opening
\title{7. Descomposició de barreja de Gaussianes}
\author{Albert Ribes}

\begin{document}

\maketitle

% \begin{abstract}
%
% \end{abstract}

% \section{}

\textbf{
Considereu el model de barreja de Gaussianes:
}

\begin{equation*}
p(x) = \sum_{k = 1}^{K} \pi_k \mathcal{N}(x; \mu_k, \Sigma_k)
\end{equation*}

\textbf{
A classe hem vist que podem treballar amb un vector de variables (anomenades latents) $z$ , on $z_i \in \{0,1\}$
i $\sum_{k = 1}^{K} z_i = 1$, de manera que $p(z_k = 1) = \pi_k$. Demostrar la descomposició alternativa de la barreja:
}

\begin{equation*}
p(x) = \sum_{z} p(z)p(x|z)
\end{equation*}

\textbf{
on $z$ es mou per tots els vectors que ténen una sola component a $1$ (i la resta a $0$).
}
\end{document}
