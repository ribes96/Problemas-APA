% Solamente para hacer la versión móbil
\documentclass[a5paper]{article}
\usepackage[margin=5mm]{geometry}
\usepackage{tgheros}
\usepackage[T1]{fontenc}
\renewcommand*\familydefault{\sfdefault}
%%%%%%%%%%%%%%%%%%%%%



% \documentclass[a4paper,10pt]{article}
\usepackage[utf8]{inputenc}
\usepackage[catalan]{babel}
\usepackage{amsmath}
\usepackage{amsfonts}
\usepackage{listings}
\usepackage[svgnames]{xcolor}
\usepackage{blkarray}
\usepackage{breqn}


%opening
\title{9. Simplifcació de la barreja de Gaussianes 1 [G]}
\author{Albert Ribes}

\begin{document}

\maketitle

% \begin{abstract}
%
% \end{abstract}

% \section{}

\textbf{
Considereu el model de barreja de Gaussianes:
}

\begin{equation*}
p(x) = \sum_{k = 1}^{K} \pi_k \mathcal{N}(x; \mu_k, \Sigma_k)
\end{equation*}

\textbf{
Preneu el cas que totes les matrius de covariança són iguals i diagonals, és a dir, $\Sigma_1 = \dots, \Sigma_K = \Sigma = diag(\sigma_1^2,\dots, \sigma_d^2)$
}

\begin{enumerate}
\item \textbf{Enraoneu en quin sentit representa una simplifcació respecte al cas general (amb matrius de
covariança generals), des dels punts de vista estadístic i geomètric.
}

Significa que las dimensiones son independientes entre ellas.

Geométricamente esto significa que cada uno de los clusters generará instancias formando una probabilidad más circular, y menos ovalada.
\item \textbf{Expresseu la funció de densitat de probabilitat $\mathcal{N}(x; \mu_k, \Sigma_k)$ que en resulta.
}

En el caso genérico la función de densidad de la distribución gaussiana multivariada con $D$ variables es:

\begin{equation*}
\mathcal{N}(x|\mu,\Sigma) = \frac{1}{(2\pi)^{(\frac{D}{2})}} \cdot \frac{1}{(|\Sigma|)^{\frac{1}{2}}} exp\Bigg( -\frac{1}{2} (x - \mu)^T\Sigma^{-1}(x - \mu) \Bigg)
\end{equation*}

Pero se puede simplificar con las siguientes propiedades:
\begin{itemize}
\item $det( diag(\alpha_1,\alpha_2, \dots, \alpha_n) ) = \prod_{i = 1}^n \alpha_i$
\item $(\alpha_1,\alpha_2, \dots, \alpha_n)^T \cdot diag(\beta_1,\beta_2, \dots, beta_n) = (\alpha_1\beta_1, \alpha_2\beta_2, \dots, \alpha_n\beta_n)^T$
\end{itemize}

Y entonces la fórmula anterior queda:

\begin{equation*}
\mathcal{N}(x|\mu,\Sigma) = \frac{1}{(2\pi)^{(\frac{D}{2})}} \cdot \frac{1}{(\prod_{i = 1}^n \sigma_i)} exp\Bigg( -\frac{1}{2} \sum_{i = 1}^{d} \Big( \sigma_i^2(x_i - \mu_i)^2 \Big) \Bigg)
\end{equation*}
\item \textbf{Construïu la funció de log-versemblança negativa.
}

En el caso de que se disponga de $M$ puntos de la distribución:
\begin{equation*}
l = -\ln \prod_{j = 1}^{M} p(x_j) 
\end{equation*}

\begin{equation*}
l = -\sum_{j = 1}^{M} \ln(p(x)) 
\end{equation*}

\begin{equation*}
l = -\sum_{j = 1}^{M} \Bigg[ 
\ln\Big(
\sum_{k = 1}^{K} \pi_k \mathcal{N}(x; \mu_k, \Sigma_k)
\Big)
\Bigg] 
\end{equation*}


\begin{equation*}
l = -\sum_{j = 1}^{M} \Bigg[ 
\ln\Big(
\sum_{k = 1}^{K}
\big(
    \pi_k \frac{1}{(2\pi)^{(\frac{D}{2})}} \cdot \frac{1}{(\prod_{i = 1}^n \sigma_i)} exp\Bigg( -\frac{1}{2} \sum_{i = 1}^{d} \Big( \sigma_i^2(x_i - \mu)^2 \Big) \Bigg)
\big)
\Big)
\Bigg] 
\end{equation*}

\item \textbf{Deriveu les equacions de l'algorisme E-M que en resulta i escriviu l'algorisme de clustering complet.
}
\item \textbf{Enraoneu sobre les implicacions (possibles avantatges/inconvenients) que representa la simplifcació
respecte el cas general des del punt de vista del clustering.
}
\end{enumerate}
\end{document}
