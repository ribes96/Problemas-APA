\documentclass[a4paper,10pt]{article}
\usepackage[utf8]{inputenc}
\usepackage[catalan]{babel}
\usepackage{amsmath}
\usepackage{amsfonts}

%opening
\title{1. L'anàlisi de components principals en dues variables}
\author{Albert Ribes}

\begin{document}

\maketitle

% \begin{abstract}
%
% \end{abstract}

% \section{}

\textbf{
Siguin $X_1$ i $X_2$ dues variables aleatòries estandarditzades i amb correlació $\rho > 0$. Construirem un PCA pas a pas a partir de la matriu de correlació teòrica $R$. Es demana:
}

\begin{enumerate}
  \item \textbf{Expresseu els dos valors propis $\lambda_1$ i $  \lambda_2$ de $R$}

  Sabemos que R tendrá 1's es la diagonal principal (pues la correlación entre una variable y ella misma siempre es 1) y, puesto que es simétrica, en las otras dos posiciones tendrá $\rho$. Entonces tenemos que:

  \begin{equation*}
    R =
    \begin{bmatrix}
      1 & \rho \\
      \rho & 1
    \end{bmatrix}
  \end{equation*}

  Si $a$ es un vector propio de $R$, hay que encontrar $\lambda$ que cumpla que:

  \begin{equation*}
    R \cdot a = \lambda \cdot a
  \end{equation*}

  \begin{equation*}
    \begin{bmatrix}
      1 & \rho \\
      \rho & 1
    \end{bmatrix}
    \begin{bmatrix}
        a_1 \\
        a_2
      \end{bmatrix}
      = \lambda \cdot
      \begin{bmatrix}
          a_1 \\
          a_2
        \end{bmatrix}
  \end{equation*}

  \begin{equation*}
    \begin{cases}
      a_1 + a_2\rho = a_1\lambda \\
      a_1\rho + a_2 = a_2\lambda
    \end{cases}
  \end{equation*}

  \begin{equation*}
    \begin{cases}
      a_2\rho = a_1(\lambda - 1) \\
      a_1\rho = a_2(\lambda - 1)
    \end{cases}
  \end{equation*}

  \begin{equation*}
    \begin{cases}
      a_1 = \frac{a_2\rho}{\lambda - 1} \\
      a_1 = \frac{a_2(\lambda - 1)}{\rho}
    \end{cases}
  \end{equation*}

  \begin{equation*}
      \frac{a_2\rho}{\lambda - 1} =
      \frac{a_2(\lambda - 1)}{\rho}
  \end{equation*}

  \begin{equation*}
    \rho^2 = (\lambda -1)^2
  \end{equation*}

  \begin{equation*}
    \begin{cases}
      \rho = \lambda_1 - 1 \\
      \rho = -(\lambda_2 - 1)
    \end{cases}
  \end{equation*}

  \begin{equation*}
    \begin{cases}
      \lambda_1 = \rho + 1 \\
      \lambda_2 = 1 - \rho
    \end{cases}
  \end{equation*}

  Y puesto que $\rho > 0$ esta claro que $\lambda_1 > \lambda_2$


  % Hay que encontrar 2 vectores $W_1$ y $W_2$ que cumplirán que $Y_1 = W_1^T \cdot \ R$ y $Y_2 = W_2^T \cdot \ R$ y que maximicen $Var(Y_1)$ y $Var(Y_2)$.
  %
  % Puesto que queremos una solución única establecemos la condición de que $||W_i||^2 = 1$
  %
  % $Var(Y_i) = Var(W_i^T \cdot R)$
  %
  % Se sabe que la varianza de un vector $W$ por una matriz $R$ se corresponde con $W^T \cdot Var() \cdot W$


  \item \textbf{Expresseu els dos vectors propis corresponents $a_1$ i $a_2$}

  Aprovechando las ecuaciones previas:

  Para encontrar $a_1$:

  \begin{equation*}
    \begin{cases}
      a_{11} = \frac{a_{12}\rho}{\lambda_1 - 1} \\
      a_{11} = \frac{a_{12}(\lambda_1 - 1)}{\rho}
    \end{cases}
  \end{equation*}

  \begin{equation*}
    \begin{cases}
      a_{11} = \frac{a_{12}\rho}{(\rho + 1) - 1} \\
      a_{11} = \frac{a_{12}((\rho + 1) - 1)}{\rho}
    \end{cases}
  \end{equation*}

  \begin{equation*}
    \begin{cases}
      a_{11} = \frac{a_{12}\rho}{\rho + 0} \\
      a_{11} = \frac{a_{12}(\rho + 0)}{\rho}
    \end{cases}
  \end{equation*}

  \begin{equation*}
    \begin{cases}
      a_{11} = a_{12} \\
      a_{11} = a_{12}
    \end{cases}
  \end{equation*}

  Y para encontrar $a_2$:

  \begin{equation*}
    \begin{cases}
      a_{21} = \frac{a_{22}\rho}{\lambda_2 - 1} \\
      a_{21} = \frac{a_{22}(\lambda_2 - 1)}{\rho}
    \end{cases}
  \end{equation*}

  \begin{equation*}
    \begin{cases}
      a_{21} = \frac{a_{22}\rho}{(1 - \rho) - 1} \\
      a_{21} = \frac{a_{22}((1 - \rho) - 1)}{\rho}
    \end{cases}
  \end{equation*}

  \begin{equation*}
    \begin{cases}
      a_{21} = \frac{a_{22}\rho}{-\rho} \\
      a_{21} = \frac{a_{22}(-\rho)}{\rho}
    \end{cases}
  \end{equation*}

  \begin{equation*}
    \begin{cases}
      a_{21} = -a_{22} \\
      a_{21} = -a_{22}
    \end{cases}
  \end{equation*}

  Entonces la respuesta es:
  \begin{equation*}
    a_1 =
    \begin{bmatrix}
      z \\ z
    \end{bmatrix}
    , z \in \mathbb{R}
  \end{equation*}

  \begin{equation*}
    a_2 =
    \begin{bmatrix}
      z \\ -z
    \end{bmatrix}
    , z \in \mathbb{R}
  \end{equation*}

  \item \textbf{Expresseu els nous eixos de coordenades, és a dir, doneu les dues components principals $Y_1$ i $Y_2$}

  No nos sirven todas las ``instancias'' de $a_1$ y $a_2$ como vectores de proyección, pues hemos establecido la condición $||a_i||^2 = 1 \Rightarrow a_{i1}^2 + a_{i_2}^2 = 1$.

  El vector de proyección de $Y_1$ debe cumplir:
  \begin{equation*}
    z^2 + z^2 = 1
  \end{equation*}

  \begin{equation*}
    2z^2 = 1
  \end{equation*}
  \begin{equation*}
    z = \frac{1}{\sqrt{2}}
  \end{equation*}

  \begin{equation*}
    z = \frac{\sqrt{2}}{2}
  \end{equation*}
\end{enumerate}

Y por lo tanto:
\begin{equation*}
  a_1 =
  \begin{bmatrix}
    \frac{\sqrt{2}}{2} \\ \\
    \frac{\sqrt{2}}{2}
  \end{bmatrix}
\end{equation*}

Y el vector de proyección de $Y_2$ debe cumplir:
\begin{equation*}
  z^2 + z^2 = 1
\end{equation*}

La norma de $a_2$ es la misma, por lo tanto, para cumplir la condición:

\begin{equation*}
  a_2 =
  \begin{bmatrix}
    \frac{\sqrt{2}}{2} \\ \\
    -\frac{\sqrt{2}}{2}
  \end{bmatrix}
\end{equation*}

Entonces:
\begin{equation*}
Y_i = a_i^T \cdot X
\end{equation*}


\begin{equation*}
Y_1 =
\begin{bmatrix}
\frac{\sqrt{2}}{2}, & \frac{\sqrt{2}}{2}
\end{bmatrix}
\cdot X
\end{equation*}


\begin{equation*}
Y_2 =
\begin{bmatrix}
\frac{\sqrt{2}}{2}, & -\frac{\sqrt{2}}{2}
\end{bmatrix}
\cdot X
\end{equation*}

Donde en este caso $X = \{X_1, X_2\}^T$

\end{document}
